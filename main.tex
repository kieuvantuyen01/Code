\documentclass[a4paper,11pt]{book}

% TEMPLATE
% ----------------------------------------------------------------------
\usepackage[utf8]{inputenc}
\usepackage[utf8]{vietnam}
\usepackage{geometry}
\geometry{
	a4paper,
	left=20mm,
	top=30mm,
	bottom=30mm,
}
\usepackage[hidelinks]{hyperref}
\usepackage{titlesec}
\usepackage{etoolbox}
\usepackage{chngcntr}
\usepackage{graphicx}
\usepackage{lipsum}
\usepackage{xcolor}
\usepackage{amsmath,amssymb,amsfonts}
\usepackage{fancyhdr}
\usepackage{booktabs}

\counterwithout{figure}{chapter}
\renewcommand\thesection{\arabic{section}}
\patchcmd{\thebibliography}{\chapter*}{\section*}{}{}
\renewcommand{\bibname}{References}
\bibliographystyle{plainnat}
\pagestyle{empty}
\titleformat{\chapter}[frame]{\normalfont}{}{10pt}{\LARGE\bfseries\filcenter}
\pagestyle{fancy}
\fancyhf{}
\pagestyle{fancy}
\fancyhead[LE,RO]{}
\fancyhead[RE,LO]{Modelling for Engineering \& Human Behaviour 2024}
\fancyfoot[C]{\thepage}

\renewcommand{\theequation}{\arabic{equation}}
\renewcommand{\thefigure}{\arabic{figure}}  
\renewcommand{\thetable}{\arabic{table}} 
% ----------------------------------------------------------------------


\begin{document}

\chapter{Công thức giải quyết bài toán StripPacking}

\thispagestyle{empty}
\vspace{-0.5cm}
\begin{center}
\begin{large}
		A. Author$^\flat$,\footnote{correspondingauthor@mail.com}
		B. Author$^\natural$
		and C. Author$^\flat$\\
		\vspace{0.25cm}
		\normalsize ($\flat$)\; I.U. de Matemática Multidisciplinar, Universitat Politècnica de València \\
        \normalsize Camí de Vera s/n, València, Spain.\\ 
        \vspace{0.1cm}
		\normalsize ($\natural$)\;  Second Author Affiliation,\\
        \normalsize Second Author Address.\\ 
\end{large}
\end{center}
\vspace{-0.5cm}

\section{Introduction}

% In this section, we explain how to translate a 2OPP into a SAT problem with order encoding. Let ri,rj ∈ R (i ̸= j) be two rectangles in a 2OPP. Let e and f be any integer. Then, the SAT encoding of a 2OPP uses four kinds of atoms, lri,j, udi,j, pxi,e, and pyi,f. lri,j is true if ri are placed at the left to the rj. udi,j is true if ri are placed at the downward to the rj. pxi,e is true if ri are placed at less than or equal to e. pyi,f is true if ri are placed at less than or equal to f. Then, inputs and constraints of a 2OPP can be encoded into a SAT problem as follows.
% For each rectangle ri, and integer e and f such that 0 ≤ e < W −wi and 0 ≤ f < H − hi, we have the 2-literal axiom clauses due to order encoding,
% ¬pxi,e ∨ pxi,e+1
% (3)
% ¬pyi,f ∨pyi,f+1
% For each rectangles ri,rj (i < j), we have the following 4-literal clauses as
% the non-overlapping constraints (2):
% lri,j ∨ lrj,i ∨ udi,j ∨ udj,i (4)
% For each rectangles ri,rj (i < j), and integer e and f such that 0 ≤ e < W − wi and 0 ≤ f < H − hj, we also have the following 3-literal clauses as the non-overlapping constraints (2):
% ¬lri,j ∨ pxi,e ∨ ¬pxj,e+wi
% ¬lrj,i ∨ pxj,e ∨ ¬pxi,e+wj ¬udi,j ∨pyi,f ∨¬pyj,f+hi ¬udj,i ∨ pyj,f ∨ ¬pyi,f +hj
% (5)
% Example. Consider the simple example of 2OPP shown in Figure 1a. We are given four rectangles (w1,h1) = (1,2),(w2,h2) = (1,2),(w3,h3) = (2, 1), (w4, h4) = (1, 1) and a strip (W, H) = (4, 4). We obtain the SAT- encoded 2OPP shown in Figure 2. This SAT problem is satisfiable and the figure of packed rectangles corresponding to a model is shown in Fig. 1b. In this case, Boolean variables of the SAT problem are assigned as follows.
% px1,0 =F, px1,1 =T, px2,0 =T, px3,1 =F, px3,2 =T, px4,1 =F, px4,2 =T py1,0 =T, py2,0 =T, py3,0 =F, py3,1 =T, py4,0 =T

% Variables
% px1,0,...,px1,3 py1,0,...,py1,3 px2,0,...,px2,2 py2,0,...,py2,3 px3,0,...,px3,3 py3,0,...,py3,2 px4,0,...,px4,3 py4,0,py4,1
% Order Constraint (3)
% ¬px1,0 ∨ px1,1, ¬px1,1 ∨ px1,2, ¬px1,2 ∨ px1,3
% ...
% ¬py4,0 ∨ py4,1, ¬py4,1 ∨ py4,2, ¬py4,2 ∨ py4,3
% Non-overlapping Constraint (4), (5) lr1,2 ∨ lr2,1 ∨ ud1,2 ∨ ud2,1
% ...
% lr3,4 ∨ lr4,3 ∨ ud3,4 ∨ ud4,3

In this section, we explain how to translate a 2OPP into a SAT problem with order encoding. Let $r_i,r_j \in R$ ($i \neq j$) be two rectangles in a 2OPP. Let $e$ and $f$ be any integer. Then, the SAT encoding of a 2OPP uses four kinds of atoms, $l_{r_i,r_j}$, $u_{r_i,r_j}$, $p_{x_i,e}$, and $p_{y_i,f}$. $l_{r_i,r_j}$ is true if $r_i$ are placed at the left to the $r_j$. $u_{r_i,r_j}$ is true if $r_i$ are placed at the downward to the $r_j$. $p_{x_i,e}$ is true if $r_i$ are placed at less than or equal to $e$. $p_{y_i,f}$ is true if $r_i$ are placed at less than or equal to $f$. Then, inputs and constraints of a 2OPP can be encoded into a SAT problem as follows.
For each rectangle $r_i$, and integer $e$ and $f$ such that $0 \leq e < W - w_i$ and $0 \leq f < H - h_i$, we have the 2-literal axiom clauses due to order encoding,
\begin{equation}
	\begin{aligned}
		&\neg p_{x_i,e} \lor p_{x_i,e+1} \\
		&\neg p_{y_i,f} \lor p_{y_i,f+1}
	\end{aligned}
\end{equation}
For each rectangles $r_i,r_j$ ($i < j$), we have the following 4-literal clauses as the non-overlapping constraints (2):
\begin{equation}
	\begin{aligned}
		&l_{r_i,r_j} \lor l_{r_j,i} \lor u_{r_i,r_j} \lor u_{r_j,i}
	\end{aligned}
\end{equation}
For each rectangles $r_i,r_j$ ($i < j$), and integer $e$ and $f$ such that $0 \leq e < W - w_i$ and $0 \leq f < H - h_j$, we also have the following 3-literal clauses as the non-overlapping constraints (2):
\begin{equation}
	\begin{aligned}
		&\neg l_{r_i,r_j} \lor p_{x_i,e} \lor \neg p_{x_j,e+w_i} \\
		&\neg l_{r_j,i} \lor p_{x_j,e} \lor \neg p_{x_i,e+w_j} \\
		&\neg u_{r_i,r_j} \lor p_{y_i,f} \lor \neg p_{y_j,f+h_i} \\
		&\neg u_{r_j,i} \lor p_{y_j,f} \lor \neg p_{y_i,f+h_j}
	\end{aligned}
\end{equation}

\section{Example}

Consider the simple example of 2OPP shown in Figure 1a. We are given four rectangles $(w_1,h_1) = (1,2),(w_2,h_2) = (1,2),(w_3,h_3) = (2, 1), (w_4, h_4) = (1, 1)$ and a strip $(W, H) = (4, 4)$. We obtain the SAT- encoded 2OPP shown in Figure 2. This SAT problem is satisfiable and the figure of packed rectangles corresponding to a model is shown in Fig. 1b. In this case, Boolean variables of the SAT problem are assigned as follows.
\begin{equation}
	\begin{aligned}
		& p_{x_1,0} = F, p_{x_1,1} = T, p_{x_2,0} = T, p_{x_3,1} = F, p_{x_3,2} = T, p_{x_4,1} = F, p_{x_4,2} = T \\
		& p_{y_1,0} = T, p_{y_2,0} = T, p_{y_3,0} = F, p_{y_3,1} = T, p_{y_4,0} = T
	\end{aligned}
\end{equation}

\section{Variables}

\begin{equation}
	\begin{aligned}
		& p_{x_1,0},...,p_{x_1,3} \quad p_{y_1,0},...,p_{y_1,3} \\
		& p_{x_2,0},...,p_{x_2,2} \quad p_{y_2,0},...,p_{y_2,3} \\
		& p_{x_3,0},...,p_{x_3,3} \quad p_{y_3,0},...,p_{y_3,2} \\
		& p_{x_4,0},...,p_{x_4,3} \quad p_{y_4,0},p_{y_4,1}
	\end{aligned}
\end{equation}

\section{Order Constraint}

\begin{equation}
	\begin{aligned}
		& \neg p_{x_1,0} \lor p_{x_1,1}, \neg p_{x_1,1} \lor p_{x_1,2}, \neg p_{x_1,2} \lor p_{x_1,3} \\
		% ...
		&...\\
		& \neg p_{y_4,0} \lor p_{y_4,1}, \neg p_{y_4,1} \lor p_{y_4,2}, \neg p_{y_4,2} \lor p_{y_4,3}
	\end{aligned}
\end{equation}

\section{Non-overlapping Constraint (1)}

\begin{equation}
	\begin{aligned}
		& l_{r_1,r_2} \lor l_{r_2,r_1} \lor u_{r_1,r_2} \lor u_{r_2,r_1} \\
		% ...
		&...\\
		& l_{r_3,r_4} \lor l_{r_4,r_3} \lor u_{r_3,r_4} \lor u_{r_4,r_3}
	\end{aligned}
\end{equation}

\section{Non-overlapping Constraint (2)}

\begin{equation}
	\begin{aligned}
		% Pair (r1,r2)
		& \neg l_{r_1,r_2} \lor p_{x_1,0} \lor \neg p_{x_2,1} \\
		& \neg l_{r_2,r_1} \lor p_{x_2,0} \lor \neg p_{x_1,1} \\
		& \neg u_{r_1,r_2} \lor p_{y_1,0} \lor \neg p_{y_2,2} \\
		& \neg u_{r_2,r_1} \lor p_{y_2,0} \lor \neg p_{y_1,2} \\
		\\
		% Pair (r1,r3)
		& \neg l_{r_1,r_3} \lor p_{x_1,0} \lor \neg p_{x_3,2} \\
		& \neg l_{r_3,r_1} \lor p_{x_3,0} \lor \neg p_{x_1,2} \\
		& \neg u_{r_1,r_3} \lor p_{y_1,0} \lor \neg p_{y_3,1} \\
		& \neg u_{r_3,r_1} \lor p_{y_3,0} \lor \neg p_{y_1,1} \\
		\\
		% Pair (r1,r4)
		& \neg l_{r_1,r_4} \lor p_{x_1,0} \lor \neg p_{x_4,2} \\
		& \neg l_{r_4,r_1} \lor p_{x_4,0} \lor \neg p_{x_1,2} \\
		& \neg u_{r_1,r_4} \lor p_{y_1,0} \lor \neg p_{y_4,1} \\
		& \neg u_{r_4,r_1} \lor p_{y_4,0} \lor \neg p_{y_1,1} \\
		\\
		% Pair (r2,r3)
		& \neg l_{r_2,r_3} \lor p_{x_2,0} \lor \neg p_{x_3,1} \\
		& \neg l_{r_3,r_2} \lor p_{x_3,0} \lor \neg p_{x_2,1} \\
		& \neg u_{r_2,r_3} \lor p_{y_2,0} \lor \neg p_{y_3,2} \\
		& \neg u_{r_3,r_2} \lor p_{y_3,0} \lor \neg p_{y_2,2} \\
		\\
		% Pair (r2,r4)
		& \neg l_{r_2,r_4} \lor p_{x_2,0} \lor \neg p_{x_4,1} \\
		& \neg l_{r_4,r_2} \lor p_{x_4,0} \lor \neg p_{x_2,1} \\
		& \neg u_{r_2,r_4} \lor p_{y_2,0} \lor \neg p_{y_4,2} \\
		& \neg u_{r_4,r_2} \lor p_{y_4,0} \lor \neg p_{y_2,2} \\
		\\
		% Pair (r3,r4)
		& \neg l_{r_3,r_4} \lor p_{x_3,0} \lor \neg p_{x_4,1} \\
		& \neg l_{r_4,r_3} \lor p_{x_4,0} \lor \neg p_{x_3,1} \\
		& \neg u_{r_3,r_4} \lor p_{y_3,0} \lor \neg p_{y_4,1} \\
		& \neg u_{r_4,r_3} \lor p_{y_4,0} \lor \neg p_{y_3,1}
	\end{aligned}
\end{equation}




\end{document}